% !TEX root = thesis.tex

\newpage
\chapter*{Abstract}

\addcontentsline{toc}{chapter}{Abstract}

This thesis describes a study and a web tool for the identification and classification of bots, according to the potential harm or threat they may cause to humans in online conversations. The focus is on Twitter, as social network platform, and bots are meant as algorithmically driven accounts that act like humans in interactions.
The problem has so far been addressed with a special attention to the detection of such automated entities among legitimate ones. The method and tool described in this paper propose a finer level of granularity and show that it is further possible to classify bots also into potential types of harm --  bots with adult content (\textit{NSFW}), bots propagating news and potential misinformation (\textit{News-Spreaders}), bots that spam product sails or job offers (\textit{Spam-Bots}) and bots who mimic interests (\textit{Fake-Followers}) -- in function of the content they share and their behaviour.
The method involves a predictive pipeline algorithm, composed by a first binary machine learning classifier, aimed to perform the detection of bots among genuine accounts (with 0.95 of AUC score) and a multi-class ensemble classifier, which goes deeper inside those aforementioned categories and classifies the potential harm (with an average F1 measure of 98\%).
The tool implementing the method, BotBuster, is a web application running the prediction script, and it is accessible for free.





\newpage
\chapter*{Sommario}

\addcontentsline{toc}{chapter}{Sommario}

Here goes the translation into Italian of the abstract. If the thesis is written in Italian, no translation into English is needed. Hence, one of the following must be checked:

\begin{itemize}
\item[\Square] Thesis written in \emph{English}, properly proofread translation needed
\item[\Square] Thesis written in \emph{Italian}, no translation needed, chapter omitted
\end{itemize}
