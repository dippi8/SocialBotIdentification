% !TEX root = ../thesis.tex
\chapter{Model selection}
\label{capitolo5}
\thispagestyle{empty}

In this chapter we will show the choices and stages behind the final model.
Starting from baseline models, we enhanced the chosen model with handcrafted features from the last chapter.\\
We saw and studied the performance improvements with validation approaches, and this phase led us to our current solution.\\
The result involves three models: two Random Forests, working on different purposes, and a Naive Bayes classifier.\\
The first Random Forest classifier has been used to provide an early filter on the separation between Genuine accounts and Bots.\\
The second Random Forest gives a classification among the five studied categories.\\
Then, a Naive Bayes classifier has been used over the same classes of the second Random Forest, but it reads and labels the users, according on their tweets only.\\
All the above mentioned algorithms were combined with a stacking ensemble methods, after considering different possibilities.

\section{Baselines}
The choices explained in this section were made at the same time of the ones listed in the Baseline section of the last chapter.\\
This is, basically, the same stage of the above-mentioned, but in a model-driven perspective.
The features involved are the ones described in that chapter, but we started from that base, to try deifferent classifers over it.
Each classifier has been fitted with the entire dataset, but considering obly baselines features.
Furhtermore, no parameters tuning has been applied, in order to minimize the results of our baslines classifier, with their standard assets.
\subsection{Random Forest}
The Random Forest was tested with the Scikit-Learn library, wit the following method and asset:\\
\textit{RandomForestClassifier(n\_estimators = 10, criterion = 'entropy', random\_state = 42)}\\
At first, it was tested with a holdout approach, with 75\% of the sample in the training set, and the 25\% in the test set.\\
The resulting confusion matrix is the following:\\

\subsection{Logistic Regression}
\subsection{KNN}
\subsection{Support Vector Machine}
\subsection{Comparison and baseline selection}
\section{Bots vs Genuines}
\section{Multiclass classifier}
\section{Text classifier}
\section{Stacking meta-classifier}
\section{Validation}